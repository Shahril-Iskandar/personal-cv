\documentclass[../main.tex]{subfiles}
\begin{document}

\section{Research Experience}
  \vspace{2pt}
  \resumeSubHeadingListStart

    \resumeSubheading
      {Research Assistant, Nanyang Technological University}{}
      {Sports Biomechanics Lab | Principal Investigators: Phillis Teng, PhD \& Swarup Mukherjee, PhD}{Aug 2023 - Present}
        \resumeItemListStart
            \resumeItem{Assisting in a project titled, "Screening and Biomechanical Risk Factors for Early Knee Osteoarthritis".}
            \resumeItem{The project aims to develop a novel screening method to identify biomechanical markers for recreational runners identified to be at risk of early knee osteoarthritis, by utilizing various equipment such as foot pressure mapping, ultrasound, tensiomyography, DEXA and markerless motion capture.}
            % \resumeItem{Facilitating recruitment of participants and data analysis.}
        \resumeItemListEnd

% This research aims to evaluate the biomechanical differences between low and high knee OA risk recreational long-distance runners. A novel screening method for runners with high knee OA risks will also be investigated.
% Summarise the literature search and identify key findings of the previous research relevant to the objectives and approach of the present study.
% Set up and be familiarized with the study equipment, assist the Principal Investigator (PI) in the planning of study processes and developing of analysis codes required by the study
% Assist in the participant recruitment process
% Coordinate the study sessions and assist in data collection
% Gather preliminary and final data.
% Organise the collected data.
% Assist the PI in data analysis
% Summarise the key findings.
% Assist the PI with writing reports and manuscripts.
% Assist the PI with organising seminars, workshops and talks.
% Other duties as assigned by the PI
  
    \resumeSubheading % “Data-driven gait rehabilitation of amputees”
      {Research Intern, Agency for Science, Technology and Research, Bioinformatics Institute}{}
      {Biophysical Modelling Lab | Principal Investigator: Chiam Keng Hwee, PhD}{Jan 2023 - Jun 2023}
        \resumeItemListStart
            \resumeItem{Contributed to a project on data-driven gait rehabilitation of lower limb amputees.}
            \resumeItem{Implemented generative AI (Stable Diffusion) to enhance existing open-source pose estimation algorithm (OpenPose \& DeepLabCut) in identifying lower limb amputee's anatomical landmarks for the purpose of gait analysis [\textcolor{blue}{S\ref{article: diffusion_openpose}}].}
            % \resumeItem{Wrote a Python script to perform data processing and analysis [\href{https://github.com/Shahril-Iskandar/ProjectAmputee}{\faGithub}].}
            % \resumeItem{Related publication: P\ref{article: astar internship}}
        \resumeItemListEnd
    
    \resumeSubheading
      {Undergraduate Research Assistant, Nanyang Technological University}{}
      {Sports Biomechanics Lab | Principal Investigator: Kong Pui Wah, PhD}{Apr 2022 - May 2023}
        \resumeItemListStart
            \resumeItem{Contributed to a project analyzing the biomechanical effects of exoskeletal in military personnel.}
            \resumeItem{Coordinated synchronized gait data collections using VICON on Bertec split-belt instrumented treadmill, Delsys EMG system and loadsol® sensors.}
            \resumeItem{Wrote MATLAB and Python scripts to extract data and conduct data analysis of ground reaction forces using statistical parametric mapping (SPM) Python package, \href{https://spm1d.org/}{spm1d} [\href{https://github.com/Shahril-Iskandar/publication-loadsol-validation-military}{\faGithub}].}
            \resumeItem{{Presented findings at an academic conference [\textcolor{blue}{CP\ref{conference_pre: assb2023}}] and co-authored a journal article [\textcolor{blue}{J\ref{article: sensors validation exoskeletal}}]}.}
        \resumeItemListEnd

    % \newpage

    \resumeSubheading
      {Honors Thesis (Grade: A+), Nanyang Technological University}{}
      {Sports Biomechanics Lab | Principal Investigators: John Cher Chay Tan, PhD \& Sofyan Sahrom, PhD}{Apr 2022 - Nov 2022}
        \resumeItemListStart
            \resumeItem{Collaborated with National Youth Sports Institute and Singapore Weightlifting Federation.}
            \resumeItem{Led the application of research ethics and designed the study protocol of evaluating the validity of a velocity-based training device in weightlifting exercises using VICON 3D motion capture cameras [\textcolor{blue}{P\ref{article: fyp}}].}
            % \resumeItem{Utilized VICON 3D motion capture cameras.}
            \resumeItem{Wrote a MATLAB script to efficiently extract data from c3d files to perform Bland-Altman analysis [\href{https://github.com/Shahril-Iskandar/undergraduate-finalyearproject}{\faGithub}].}
            \resumeItem{Awarded Best Poster Presentation \href{https://drive.google.com/file/d/1mcvcseOSgvzBLQg3tZVkGyIOOxVw6gKw/view?usp=sharing}{award} at the 11$^{th}$ Lau Teng Chuan Physical Education \& Sports Science Symposium. \href{https://drive.google.com/file/d/1kLCyjZXhJKm-lV7M_6sJ5hKYgOW0xfY2/view?usp=sharing}{\faFileImageO}}
        \resumeItemListEnd

    % \newpage

    \resumeSubheading % “A Video-Based Treadmill Running Analysis Model”
      {Undergraduate Research Programme  \href{https://www.ntu.edu.sg/education/undergraduate-research-experience-on-campus-(ureca)}{(URECA)}, Nanyang Technological University}{}
      {Sports Biomechanics Lab | Principal Investigator: Kong Pui Wah, PhD}{Aug 2021 - Aug 2022}
        \resumeItemListStart       
            \resumeItem{Contributed to the development of a video-based analysis model for assessing treadmill running biomechanics.}
            \resumeItem{Facilitated over 40 participants' recruitment and utilized Kinovea to analyze running kinematics.}
            \resumeItem{Presented findings at 2 academic conferences [\textcolor{blue}{CP\ref{conference_pre: isbs2022}}, \textcolor{blue}{CP\ref{conference_pre: icur2022}}] and published 2 journal articles [\textcolor{blue}{J\ref{article: paah foot morphology}}, \textcolor{blue}{J\ref{article: frontiers foot inversion}}].}
            \resumeItem{Conferred the title “\href{https://drive.google.com/file/d/1Ut1poXJ_C8lSSeckPBVFy2DgOusebh7V/view?usp=sharing}{NTU President Research Scholar}” for completing the programme with Distinction.}
        \resumeItemListEnd

    \resumeSubheading % Dose-response of leucine on muscle maintenance during weight loss
      {Undergraduate Research Assistant, Nanyang Technological University}{}
      {Human Bioenergetics Lab | Principal Investigator: Yang Yifan, PhD}{Sep 2020 - Mar 2021}
        \resumeItemListStart
            \resumeItem{Contributed to the project assessing the dose-response of leucine on muscle maintenance during weight loss.}
            \resumeItem{Independently recruited over 20 participants and coordinated weekly anthropometric measurements.} % 18 participants
            % \resumeItem{collected key physiological and anthropometric data every week.}
            % \resumeItem{Provided clear explanations of DEXA scan results to participants.}
            \resumeItem{Verified accuracy of participant's data entry for daily physical activity, sleep, and dietary intake log.}
            % Co-ordinated participants to attend weekly data collection
        \resumeItemListEnd
        
    \resumeSubheading
      {Final-Year Thesis (Grade: A), Republic Polytechnic}{}
      {Biomechanics Lab | Principal Investigators: Shigetada Kudo, PhD \& Alexander Ong, PhD}{Apr 2016 - Oct 2016}
        \resumeItemListStart
            \resumeItem{Collaborated with Singapore Sports Institute and Singapore Swimming Association.}
            % \resumeItem{Utilized a video camera to capture 3 youth athletes’ diving movements.}
            \resumeItem{Performed 2D kinematic analysis on springboard diving using Kinovea.}
            % \resumeItem{Presented findings to the coach.}
        \resumeItemListEnd
    
  \resumeSubHeadingListEnd

\end{document}